\definecolor{HEADDER}{rgb}{0.9, 0.9, 1}
\definecolor{ODD}{gray}{1.0} 
\definecolor{EVEN}{gray}{0.9}

\definecolor{hint.info}      {rgb} {0, 0, 1}
\definecolor{hint.example}   {rgb} {0, 1, 0} 
\definecolor{hint.important} {rgb} {1, 0, 0}
\definecolor{marking}{rgb}{0.6, 0, 0}

\definecolor{prim.head} {rgb} {.6, .6, .6}
\definecolor{prim.light} {rgb} {.8, .8, .8}
\definecolor{prim.dark} {rgb} {1, 1, 1}

\definecolor{send.head} {rgb} {.5, 1, .5}
\definecolor{send.light} {rgb} {.6, 1, .6}
\definecolor{send.dark} {rgb} {.8, 1, .8}

\definecolor{recv.head} {rgb} {1, .5, .5}
\definecolor{recv.light} {rgb} {1, .6, .6}
\definecolor{recv.dark} {rgb} {1, .8, .8}

\definecolor{evnt.head} {rgb} {1, 1, .5}
\definecolor{evnt.light} {rgb} {1, 1, .6}
\definecolor{evnt.dark} {rgb} {1, 1, .8}

\definecolor{typedef.head} {rgb} {.5, .5, 1}
\definecolor{typedef.light} {rgb} {.6, .6, 1}
\definecolor{typedef.dark} {rgb} {.8, .8, 1}

\newenvironment{info}{\begin{tcolorbox}[colback=hint.info!2,colframe=hint.info!70!black,title=\textsc{Info}]}{\end{tcolorbox}}
\newenvironment{infocont}{\begin{tcolorbox}[colback=hint.info!2,colframe=hint.info!70!black,title=\textsc{Info (Cont.)}]}{\end{tcolorbox}}
\newenvironment{wichtig}{\begin{tcolorbox}[colback=hint.important!2,colframe=hint.important!70!black,title=\textsc{Important}]}{\end{tcolorbox}}
\newenvironment{beispiel}{\begin{tcolorbox}[colback=hint.example!2,colframe=hint.example!70!black,title=\textsc{Example}]}{\end{tcolorbox}}

\newenvironment{beispielcont}{\begin{tcolorbox}[colback=hint.example!2,colframe=hint.example!70!black,title=\textsc{Example (Cont.)}]}{\end{tcolorbox}}

\newenvironment{additional}{}{}

\newcommand{\therequest}{}

\newenvironment{request}[1]{\renewcommand{\therequest}{#1}\subsubsection{#1}}{}

\newenvironment{send}{\rowcolors{2}{send.light}{send.dark}\paragraph*{Request}\mbox{}\\
	\begin{longtable}{||p{4cm}|p{3cm}|p{7cm}||}
		\rowcolor{send.head} \hline \textbf{Name} & \textbf{Type}& \textbf{Description} \\
		\hline\hline
		\param{request}{str}{Request Type (''\texttt{\therequest}'')}
		\param{messageid}{str}{Message ID, this parameter is mirrored by the server}
	}{\end{longtable}}

\newenvironment{recv}{\rowcolors{2}{recv.light}{recv.dark}\paragraph*{Response}\mbox{}\\
	\begin{longtable}{||p{4cm}|p{3cm}|p{7cm}||}
		\rowcolor{recv.head} \hline \textbf{Name} & \textbf{Type}& \textbf{Description} \\ \hline\hline
		\param{messageid}{str}{Message ID, this parameter is mirrored by the server}}{\end{longtable}}

\newenvironment{evnt}{\rowcolors{2}{evnt.light}{evnt.dark}\paragraph*{Message}\mbox{}\\
	\begin{longtable}{||p{4cm}|p{3cm}|p{7cm}||}
		\rowcolor{evnt.head} \hline \textbf{Name} & \textbf{Type}& \textbf{Description} \\ \hline\hline
		\param{event}{str}{Name of this event}
		\param{source}{str}{Source of this event}}{\end{longtable}}

\newenvironment{type}[1]{\renewcommand{\therequest}{#1}\subsection{#1}}{}

\newenvironment{typedef}{\rowcolors{2}{typedef.light}{typedef.dark}
	\begin{longtable}{||p{4cm}|p{3cm}|p{7cm}||}
		\rowcolor{typedef.head} \hline \textbf{Name} & \textbf{Type}& \textbf{Description} \\ \hline\hline}{\end{longtable}}

\newcommand{\param}[3]{\texttt{#1} & \textsl{#2} & #3 \\\hline}
\newcommand{\primitive}[2]{\textsl{#1} & #2 \\\hline}

\newcommand{\triggers}[1]{\textbf{Triggers Event:} #1}



\newcommand{\keyword}[1]{\textbf{\color{marking}#1}}
\newcommand{\figref}[1]{Figure~\ref{#1}}
\newcommand{\todo}[1]{\textbf{\color{red}TODO: #1}}

\newcommand{\sitem}[1]{\begin{itemize}
		\item #1
\end{itemize}}

\newcommand{\plang}[1]{\textit{#1}}

\newcommand*{\pdfimg}[2]{
	{\centering
		\includegraphics[width=#1\textwidth]{fig/fig_#2.pdf}}
}

\newcommand*{\pdffig}[3]{\begin{figure}[h!]
		\centering
		\includegraphics[width=#1\textwidth]{fig/fig_#2.pdf}
		\caption{%
			#3
		}
		\label{fig:fig_#2}
\end{figure}}

\newcommand{\sidebyside}[2]{\begin{minipage}{0.45\textwidth}
		#1
	\end{minipage} \hspace{0.1\textwidth}
	\begin{minipage}{0.45\textwidth}
		#2
\end{minipage}}

\newcommand*{\pdffigs}[4]{\begin{figure}[h!]
		\centering
		\includegraphics[width=#1\textwidth]{fig/fig_#2.pdf}
		\includegraphics[width=#1\textwidth]{fig/fig_#3.pdf}
		\caption{%
			#4
		}
		\label{fig:fig_#2}
\end{figure}}

\setcounter{MaxMatrixCols}{20}
%\rowcolors{2}{EVEN}{ODD}
\newcommand{\mathbox}[1]{\text{\fbox{$\displaystyle#1$}}}

%#################################
%#####      LSTLISTINGS      #####
%#################################


\lstset{ 
	backgroundcolor=\color{white},   % choose the background color; you must add \usepackage{color} or \usepackage{xcolor}; should come as last argument
	basicstyle=\footnotesize,        % the size of the fonts that are used for the code
	breakatwhitespace=false,         % sets if automatic breaks should only happen at whitespace
	breaklines=true,                 % sets automatic line breaking
	captionpos=b,                    % sets the caption-position to bottom
	commentstyle=\color{green!60!black},      % comment style
	deletekeywords={...},            % if you want to delete keywords from the given language
	escapeinside={\%*}{*)},          % if you want to add LaTeX within your code
	extendedchars=true,              % lets you use non-ASCII characters; for 8-bits encodings only, does not work with UTF-8
	frame=all,	                 % adds a frame around the code
	keepspaces=true,                 % keeps spaces in text, useful for keeping indentation of code (possibly needs columns=flexible)
	keywordstyle=\color{violet},     % keyword style
	language=C,                 	 % the language of the code
	morekeywords={*,..., fi, do, od, return, then},            % if you want to add more keywords to the set
	numbers=left,                    % where to put the line-numbers; possible values are (none, left, right)
	numbersep=5pt,                   % how far the line-numbers are from the code
	numberstyle=\tiny\color{gray},   % the style that is used for the line-numbers
	rulecolor=\color{black!50},         % if not set, the frame-color may be changed on line-breaks within not-black text (e.g. comments (green here))
	showspaces=false,                % show spaces everywhere adding particular underscores; it overrides 'showstringspaces'
	showstringspaces=false,          % underline spaces within strings only
	showtabs=false,                  % show tabs within strings adding particular underscores
	stepnumber=1,                    % the step between two line-numbers. If it's 1, each line will be numbered
	stringstyle=\color{blue},        % string literal style
	tabsize=2,	                     % sets default tabsize to 2 spaces
	title=\lstname                   % show the filename of files included with \lstinputlisting; also try caption instead of title
}


\lstdefinelanguage
[x64]{Assembler}     % add a "x64" dialect of Assembler
[x86masm]{Assembler} % based on the "x86masm" dialect
% with these extra keywords:
{morekeywords={CDQE,CQO,CMPSQ,CMPXCHG16B,JRCXZ,LODSQ,MOVSXD, %
		POPFQ,PUSHFQ,SCASQ,STOSQ,IRETQ,RDTSCP,SWAPGS, %
		rax,rdx,rcx,rbx,rsi,rdi,rsp,rbp, %
		r8,r8d,r8w,r8b,r9,r9d,r9w,r9b, %
		r10,r10d,r10w,r10b,r11,r11d,r11w,r11b, %
		r12,r12d,r12w,r12b,r13,r13d,r13w,r13b, %
		r14,r14d,r14w,r14b,r15,r15d,r15w,r15b}} % etc.

\lstdefinelanguage{JavaScript}{
	keywords={break, case, catch, continue, debugger, default, delete, do, else, finally, for, function, if, in, instanceof, new, return, switch, this, throw, try, typeof, var, void, while, with},
	morecomment=[l]{//},
	morecomment=[s]{/*}{*/},
	morestring=[b]',
	morestring=[b]",
	sensitive=true
}

\lstdefinelanguage{Scheme}{
	keywords={define, lambda},
	morekeywords=[1]{define, define-syntax, define-macro, lambda, define-stream, stream-lambda},
	morekeywords=[2]{begin, call-with-current-continuation, call/cc,
		call-with-input-file, call-with-output-file, case, cond,
		do, else, for-each, if,
		let*, let, let-syntax, letrec, letrec-syntax,
		let-values, let*-values,
		and, or, not, delay, force,
		quasiquote, quote, unquote, unquote-splicing,
		map, fold, syntax, syntax-rules, eval, environment, query, car, cdr, cons, set!, set-car!, set-cdr!},
	morekeywords=[3]{import, export},
	morecomment=[l]{;},
	morecomment=[s]{\#|}{|\#},
	alsodigit=!\$\%&*+-./:<=>?@^_~,
	literate=*{`}{{`}}{1},
	sensitive=true
}