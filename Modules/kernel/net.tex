% !TeX root = ../../Covise_Notes.tex
\renewcommand{\namespace}{kernel::net::}


\newenvironment{messagetypes}{\rowcolors{2}{evnt.light}{evnt.dark}\mbox{}\\
	\begin{longtable}{||p{7.4cm}|p{.5cm}|p{6cm}||} 
		\rowcolor{evnt.head} \hline \textbf{Name} & \textbf{ID} & \textbf{Comment} \\ \hline\hline}{\end{longtable}}

\newenvironment{messagedesc}{\rowcolors{2}{evnt.light}{evnt.dark}\mbox{}\\
	\begin{longtable}{||p{3.4cm}|p{2.5cm}|p{8cm}||}
		\rowcolor{evnt.head} \hline \textbf{Name} & \textbf{Length} & \textbf{Comment} \\ \hline\hline}{\end{longtable}}

\newcommand{\messageType}[3]{\lstinline|#1| & #2 & #3 \\
	\hline}

\newcommand{\messageTypeDep}[3]{\rowcolor{linedeprecated}\lstinline|#1| & #2 & \color{red}\textbf{!!! DEPRECATED !!! DO NOT USE !!!}\color{black}, #3 \\
	\hline}

\newcommand{\paramfour}[4]{#1 & \texttt{#2} & \textsl{#3} & #4 \\\hline}

\newcommand{\dtypeBOOL}[1]{#1 & \texttt{DataType BOOL} & \textsl{byte} & The next token is to be interpreted as bool, \textbf{Value = 0x07} \\\hline}
\newcommand{\dtypeLONG}[1]{#1 & \texttt{DataType INT64} & \textsl{byte} & The next token is to be interpreted as int64, \textbf{Value = 0x08} \\\hline}
\newcommand{\dtypeINT}[1]{#1 & \texttt{DataType INT32} & \textsl{byte} & The next token is to be interpreted as int32, \textbf{Value = 0x09} \\\hline}
\newcommand{\dtypeFLOAT}[1]{#1 & \texttt{DataType FLOAT} & \textsl{byte} & The next token is to be interpreted as float, \textbf{Value = 0x0a} \\\hline}
\newcommand{\dtypeDOUBLE}[1]{#1 & \texttt{DataType DOUBLE} & \textsl{byte} & The next token is to be interpreted as double, \textbf{Value = 0x0b} \\\hline}
\newcommand{\dtypeSTRING}[1]{#1 & \texttt{DataType STRING} & \textsl{byte} & The next token is to be interpreted as string, \textbf{Value = 0x0c} \\\hline}
\newcommand{\dtypeCHAR}[1]{#1 & \texttt{DataType CHAR} & \textsl{byte} & The next token is to be interpreted as char, \textbf{Value = 0x0d} \\\hline}
\newcommand{\dtypeDATA}[1]{#1 & \texttt{DataType DATA} & \textsl{byte} & The next token is to be interpreted as DataHandle / TokenBuffer, \textbf{Value = 0x0e} \\\hline}
\newcommand{\dtypeBINARY}[1]{#1 & \texttt{DataType BINARY} & \textsl{byte} & The next token is to be interpreted as Binary Data, \textbf{Value = 0x0f} \\\hline}


\newcommand{\createStructName}[1]{struct#1}
\newcommand{\definestruct}[2]{\expandafter\def\csname\createStructName{#1}\endcsname{\rowcolor{sendstruct.subhead} \multicolumn{4}{||c||}{\textsc{Struct #1}}\\ \hline #2 \rowcolor{sendstruct.subhead} \multicolumn{4}{||c||}{\textsc{End of struct #1}}\\ \hline} \begin{structenv}{#1} \relax #2\end{structenv}}

\newcommand{\listparam}[2]{\rowcolor{sendstruct.list} \multicolumn{4}{||c||}{\textsc{List[#1]}}\\ \hline #2 \rowcolor{sendstruct.list} \multicolumn{4}{||c||}{\textsc{End of List}}\\ \hline}

%\newenvironment{tcpmessage}{\rowcolors{2}{send.light}{send.dark}\paragraph*{AAA}\mbox{}\\
%	\begin{longtable}{||p{2cm}|p{4cm}|p{3cm}|p{5cm}||}
%		\rowcolor{send.head} \hline \textbf{Index} & \textbf{Name} & \textbf{Type}& \textbf{Description} \\
%		\hline\hline
%		\paramfour{0..3}{sender}{4 byte}{}
%		\paramfour{4..7}{sender_type}{4 byte}{}
%		\paramfour{8..11}{message_type}{4 byte}{MessageType 00: AAA}
%		\paramfour{12..15}{length}{4 byte}{}
%		\hline
%	}{\end{longtable}}


\newenvironment{structenv}[2]{\rowcolors{2}{struct.light}{struct.dark} \paragraph{#1}\mbox{}\\ \begin{longtable}{||p{2cm}|p{4cm}|p{2cm}|p{6cm}||}
		\hline \rowcolor{struct.head} \textbf{Index} & \textbf{Name} & \textbf{Type}& \textbf{Description} \\
		\hline\hline
	}{\end{longtable}}

\newenvironment{tcpmessage}[2]{\rowcolors{2}{send.light}{send.dark} \paragraph{#1 (Type #2)}\mbox{}\\ \begin{longtable}{||p{2cm}|p{4cm}|p{2cm}|p{6cm}||}
		\hline \rowcolor{send.head} \textbf{Index} & \textbf{Name} & \textbf{Type}& \textbf{Description} \\
		\hline\hline
		\rowcolor{send.subhead} \multicolumn{4}{||c||}{\textbf{HEADER}} \\
		\hline
			\paramfour{0..3}{sender}{int32}{Sender ID, max 3 byte verwenden}
			\paramfour{4..7}{sender\_type}{int32}{Sender Type}
			\paramfour{8..11}{message\_type}{int32}{MessageType #2} % \newline #1
			\paramfour{12..15}{length}{int32}{Length of payload in Bytes}
			\rowcolor{send.subhead} \multicolumn{4}{||c||}{\textbf{PAYLOAD}} \\
			\hline
}{\end{longtable}}

The net folder contains basic classes related to establishing connections between client and server using sockets.

\section{Lowlevel Protocol}
\seclbl{protocol}

\subsection{Overview}

TCP Connection for regular messages, UDP Connection for UDP Messages, Optional with SSL Encryption (at least for regular messages, possibly for both). Also there has to exist a way to exchange data on the machine via shared memory.

There exist at least 2 distinct message types: Messages and UDP Messages. UDP Messages are regular messages with a stripped down header sent using the udp protocol (UDP Messages were introduced in May 2019, while regular messages exist since 1993).

The Default port used is 31000.
The Default port used for VRB is 31800 for TCP and 31801 for UDP.


\subsection{Message}
\seclbl{protocol::message}

\subsubsection{Message Format}
\seclbl{protocol::message::message-format}
Defined in \secnref{classes::message}

Each Message contains the following:

\begin{messagedesc}
	\messageType{sender}{3 byte}{Sender of message, max 3 bytes}
	\messageType{send_type}{int}{Sender Type, defaults to UNDEFINED, actual size depending on Architecture and Compiler. Should be 4 bytes on most modern systems, but could be 2. For a list of valid values, see \secnref{protocol::message::sender-types}}
	\messageType{type}{int}{Message Type, defaults to EMPTY, actual size depending on Architecture and Compiler. Should be 4 bytes on most modern systems, but could be 2. For a list of valid values, see \secnref{protocol::message::message-types}}
	\messageType{data}{}{Bytes containing custom data}
\end{messagedesc}

More specifically, the Header consists of 4 IEEE ints (16 bytes):

\begin{itemize}
	\item[[0:3]] sender
	\item[[4:7]] senderType
	\item[[8:11]] messageType
	\item[[12:15]] dataLength - Length of data in bytes
\end{itemize}



\subsubsection{Message Types}
\seclbl{protocol::message::message-types}
Defined in \secnref{classes::message-types}

\begin{messagetypes}
	\messageType{EMPTY}{-1}{Used as default in message constructor, should be ignored if send to the server (see src/sys/controller/handler.cpp, handleMsg, ln 246}
	\messageType{MSG_FAILED}{0}{Generic Failed message. Used as response when an operation failed}
	\messageType{MSG_OK}{1}{Generic success message. Used as response when an operation succeeded}
	\messageType{INIT}{2}{First message sent in communication ? \textbf{Possibly Deprecated ?} Used in src/module/renderer/VRMLRenderer, check\_aws(), ln. 498}
	\messageType{FINISHED}{3}{Finish message from a rendermodule (see src/sys/controller/handler.cpp, ln. 278), sends \lstinline|COVISE_MESSAGE_UI| with parameter ''FINISHED $\backslash$ n'' in case no modules are running anymore}
	\messageType{SEND}{4}{}
	\messageType{ALLOC}{5}{}
	\messageType{UI}{6}{UI Messages. In case body starts with ''UNDO'', this is an undo-action. For more keywords, see src/sys/controller/handler.cpp, ln. 969 ff}
	\messageType{APP_CONTACT_DM}{7}{}
	\messageType{DM_CONTACT_DM}{8}{}
	\messageType{SHM_MALLOC}{9}{}
	\messageType{SHM_MALLOC_LIST}{10}{}
	\messageType{MALLOC_OK}{11}{}
	\messageType{MALLOC_LIST_OK}{12}{}
	\messageType{MALLOC_FAILED}{13}{}
	\messageType{PREPARE_CONTACT}{14}{}
	\messageType{PREPARE_CONTACT_DM}{15}{}
	\messageType{PORT}{16}{}
	\messageType{GET_SHM_KEY}{17}{}
	\messageType{NEW_OBJECT}{18}{}
	\messageType{GET_OBJECT}{19}{}
	\messageType{REGISTER_TYPE}{20}{}
	\messageType{NEW_SDS}{21}{}
	\messageType{SEND_ID}{22}{}
	\messageType{ASK_FOR_OBJECT}{23}{}
	\messageType{OBJECT_FOUND}{24}{}
	\messageType{OBJECT_NOT_FOUND}{25}{}
	\messageType{HAS_OBJECT_CHANGED}{26}{}
	\messageType{OBJECT_UPDATE}{27}{}
	\messageType{OBJECT_TRANSFER}{28}{}
	\messageType{OBJECT_FOLLOWS}{29}{}
	\messageType{OBJECT_OK}{30}{}
	\messageType{CLOSE_SOCKET}{31}{Should be ignored if send to the server (see src/sys/controller/handler.cpp, handleMsg, ln 246), removes the client from the session in VRB Server (See kernel/vrb/server/VrbMessageHandler.cpp)}
	\messageType{DESTROY_OBJECT}{32}{}
	\messageType{CTRL_DESTROY_OBJECT}{33}{}
	\messageType{QUIT}{34}{QUIT message from user interface, opencover or other sources. Handled by server, depending on parameters. Quits the current session. (see src/sys/controller/handler.cpp), removes the client from the session in VRB Server (See kernel/vrb/server/VrbMessageHandler.cpp)}
	\messageType{START}{35}{}
	\messageType{COVISE_ERROR}{36}{Sent to all modules as \lstinline|COVISE_MESSAGE_COVISE_ERROR|. In case of error overflow, the info that there is overflow is sent instead (see src/sys/controller/handler.cpp)}
	\messageType{INOBJ}{37}{}
	\messageType{OUTOBJ}{38}{}
	\messageType{OBJECT_NO_LONGER_USED}{39}{}
	\messageType{SET_ACCESS}{40}{}
	\messageType{FINALL}{41}{Module says it has finished. Server Side resources are released, and server sends \lstinline|COVISE_MESSAGE_UI| with parameter ''FINISHED $\backslash n$'' in case no modules are running anymore (see src/sys/controller/handler.cpp)}
	\messageType{ADD_OBJECT}{42}{}
	\messageType{DELETE_OBJECT}{43}{}
	\messageType{NEW_OBJECT_VERSION}{44}{}
	\messageType{RENDER}{45}{Forwarded to all other renderers by the server (see src/sys/controller/handler.cpp), Passed to all Participants in VRB Server (See kernel/vrb/server/VrbMessageHandler.cpp)}
	\messageType{WAIT_CONTACT}{46}{}
	\messageType{PARINFO}{47}{send message to all userinterfaces (see src/sys/controller/handler.cpp)}
	\messageType{MAKE_DATA_CONNECTION}{48}{}
	\messageType{COMPLETE_DATA_CONNECTION}{49}{}
	\messageType{SHM_FREE}{50}{}
	\messageType{GET_TRANSFER_PORT}{51}{}
	\messageType{TRANSFER_PORT}{52}{}
	\messageType{CONNECT_TRANSFERMANAGER}{53}{}
	\messageType{STDINOUT_EMPTY}{54}{}
	\messageType{WARNING}{55}{Messages are relayed as Error to all renderers. Data will be modified by prefix ''WARNING'' and suffix ''$\backslash n$'' (see src/sys/controller/handler.cpp)}
	\messageType{INFO}{56}{Messages are relayed as Error to all renderers. Data will be modified by prefix ''INFO'' and suffix ''$\backslash n$'' (see src/sys/controller/handler.cpp)}
	\messageType{REPLACE_OBJECT}{57}{}
	\messageType{PLOT}{58}{Forwarded to all other renderers by the server (see src/sys/controller/handler.cpp)}
	\messageType{GET_LIST_OF_INTERFACES}{59}{}
	\messageType{USR1}{60}{}
	\messageType{USR2}{61}{}
	\messageType{USR3}{62}{}
	\messageType{USR4}{63}{}
	\messageType{NEW_OBJECT_OK}{64}{}
	\messageType{NEW_OBJECT_FAILED}{65}{}
	\messageType{NEW_OBJECT_SHM_MALLOC_LIST}{66}{}
	\messageType{REQ_UI}{67}{Relayed to all Userinterface types (see src/sys/controller/handler.cpp)}
	\messageType{NEW_PART_ADDED}{68}{}
	\messageType{SENDING_NEW_PART}{69}{}
	\messageType{FINPART}{70}{}
	\messageType{NEW_PART_AVAILABLE}{71}{}
	\messageType{OBJECT_ON_HOSTS}{72}{}
	\messageType{OBJECT_FOLLOWS_CONT}{73}{}
	\messageType{CRB_EXEC}{74}{}
	\messageType{COVISE_STOP_PIPELINE}{75}{Sets the status of the module with the name from the parameters to stopping (see src/sys/controller/handler.cpp)}
	\messageType{PREPARE_CONTACT_MODULE}{76}{}
	\messageType{MODULE_CONTACT_MODULE}{77}{}
	\messageType{SEND_APPL_PROCID}{78}{}
	\messageType{INTERFACE_LIST}{79}{}
	\messageType{MODULE_LIST}{80}{}
	\messageType{HOSTID}{81}{}
	\messageType{MODULE_STARTED}{82}{}
	\messageType{GET_USER}{83}{}
	\messageType{SOCKET_CLOSED}{84}{Should be ignored if send to the server (see src/sys/controller/handler.cpp, handleMsg, ln 246 (see src/sys/controller/handler.cpp), removes the client from the session in VRB Server (See kernel/vrb/server/VrbMessageHandler.cpp)}
	\messageType{NEW_COVISED}{85}{}
	\messageType{USER_LIST}{86}{}
	\messageType{STARTUP_INFO}{87}{}
	\messageType{CO_MODULE}{88}{}
	\messageType{WRITE_SCRIPT}{89}{}
	\messageType{CRB}{90}{}
	\messageType{GENERIC}{91}{Evaluated for Keywords, and handled accordingly. Message Body is parsed (see src/sys/controller/handler.cpp)}
	\messageType{RENDER_MODULE}{92}{Forwarded to all other renderers by the server (see src/sys/controller/handler.cpp), Passed to all Participants in VRB Server (See kernel/vrb/server/VrbMessageHandler.cpp)}
	\messageType{FEEDBACK}{93}{Messages from Renderer sent to a module. Message is parsed and sent to the specified module (see src/sys/controller/handler.cpp)}
	\messageType{VRB_CONTACT}{94}{Initialise connection from client}
	\messageType{VRB_CONNECT_TO_COVISE}{95}{Connect to covise. Message is partly parsed for an IP, rest of message is forewarded to client with the given IP}
	\messageType{END_IMM_CB}{96}{}
	\messageType{NEW_DESK}{97}{}
	\messageTypeDep{VRB_SET_USERINFO}{98}{Contains Information about clients connected to the server. Sending this message type to the server is deprecated, the server itself is still using it to send messages to clients}
	\messageType{VRB_GET_ID}{99}{Returns a new private session ID if the message originated from a connection \todo{Check, code seems not too clear}}
	\messageType{VRB_SET_GROUP}{100}{}
	\messageType{VRB_QUIT}{101}{}
	\messageType{VRB_SET_MASTER}{102}{Determines new master state and informs all clients in session about new master}
	\messageType{VRB_GUI}{103}{}
	\messageType{VRB_CLOSE_VRB_CONNECTION}{104}{}
	\messageType{VRB_REQUEST_FILE}{105}{Requests the contents of a file from the server. (See kernel/vrb/server/VrbMessageHandler.cpp)}
	\messageType{VRB_SEND_FILE}{106}{Forewards the message to the client requesting the file. MessageBody is parsed. (See kernel/vrb/server/VrbMessageHandler.cpp)}
	\messageTypeDep{VRB_CURRENT_FILE}{107}{use \lstinline|COVISE_MESSAGE_VRB_REQUEST _FILE| and sharedState \lstinline|coVRFileManager_filePaths| instead. (See kernel/vrb/server/VrbMessageHandler.cpp)}
	\messageType{CRB_QUIT}{108}{}
	\messageType{REMOVED_HOST}{109}{}
	\messageType{START_COVER_SLAVE}{110}{}
	\messageType{VRB_REGISTRY_ENTRY_CHANGED}{111}{}
	\messageType{VRB_REGISTRY_ENTRY_DELETED}{112}{}
	\messageType{VRB_REGISTRY_SUBSCRIBE_CLASS}{113}{Creates an Observer for the specified class with the specified senderID and class for the given sessionID}
	\messageType{VRB_REGISTRY_SUBSCRIBE_VARIABLE}{114}{Creates an Observer for the specified variable with the specified senderID and class for the given sessionID}
	\messageType{VRB_REGISTRY_CREATE_ENTRY}{115}{Creates a registry entry and puts the sender in the session with the given ID.}
	\messageType{VRB_REGISTRY_SET_VALUE}{116}{Set Registry Value on the server}
	\messageType{VRB_REGISTRY_DELETE_ENTRY}{117}{Deletes the specified registry entry from the specified session}
	\messageType{VRB_REGISTRY_UNSUBSCRIBE_CLASS}{118}{Stops the Observer for the specified class with the specified senderID and class for the given sessionID}
	\messageType{VRB_REGISTRY_UNSUBSCRIBE_VARIABLE}{119}{Stops the Observer for the specified variable with the specified senderID and class for the given sessionID}
	\messageType{SYNCHRONIZED_ACTION}{120}{}
	\messageType{ACCESSGRID_DAEMON}{121}{}
	\messageType{TABLET_UI}{122}{}
	\messageType{QUERY_DATA_PATH}{123}{}
	\messageType{SEND_DATA_PATH}{124}{}
	\messageType{VRB_FB_RQ}{125}{Handles FileBrowser Request}
	\messageType{VRB_FB_SET}{126}{}
	\messageType{VRB_FB_REMREQ}{127}{Handles FileBrowser remote Request}
	\messageType{UPDATE_LOADED_MAPNAME}{128}{Relayed to all Userinterface types, MAKRO (''UPDATE\_LOADED\_MAPNAME'') executed (see src/sys/controller/handler.cpp)}
	\messageType{SSLDAEMON}{129}{}
	\messageType{VISENSO_UI}{130}{}
	\messageType{PARAMDESC}{131}{}
	\messageType{VRB_REQUEST_NEW_SESSION}{132}{Creates a new (public) session for this client}
	\messageType{VRBC_SET_SESSION}{133}{}
	\messageType{VRBC_SEND_SESSIONS}{134}{Contains information about all available sessions}
	\messageType{VRBC_CHANGE_SESSION}{135}{}
	\messageType{VRBC_UNOBSERVE_SESSION}{136}{}
	\messageType{VRB_SAVE_SESSION}{137}{}
	\messageType{VRB_LOAD_SESSION}{138}{}
	\messageType{VRB_MESSAGE}{139}{Passed to all Participants in VRB Server (See kernel/vrb/server/VrbMessageHandler.cpp)}
	\messageType{VRB_PERMIT_LAUNCH}{140}{Forewards a permit-launch typed message to the client specified in the message}
	\messageType{BROADCAST_TO_PROGRAM}{141}{Broadcasts the message to the covise::Program specified in the token buffer}
	\messageType{NEW_UI}{142}{Processed and handled by server. Seems to request current collaborative state or list of partners, depending on parameters (see src/sys/controller/handler.cpp)}
	\messageType{PROXY}{143}{}
	\messageType{SOUND}{144}{}
	\messageType{LAST_DUMMY_MESSAGE}{145}{}
\end{messagetypes}

\subsubsection{Sender Types}
\seclbl{protocol::message::sender-types}

Defined in \secnref{classes::message-types}

\begin{messagetypes}
	\messageType{UNDEFINED}{0}{Used as default value in message constructor}
	\messageType{CONTROLLER}{1}{}
	\messageType{CRB}{2}{}
	\messageType{USERINTERFACE}{3}{}
	\messageType{RENDERER}{4}{}
	\messageType{APPLICATIONMODULE}{5}{}
	\messageType{TRANSFERMANAGER}{6}{}
	\messageType{SIMPLEPROCESS}{7}{}
	\messageType{SIMPLECONTROLLER}{8}{}
	\messageType{STDINOUT}{9}{}
	\messageType{COVISED}{10}{}
	\messageType{VRB}{11}{}
	\messageType{SENDER_SOUND}{12}{}
	\messageType{ANY}{}{}
\end{messagetypes}

\subsubsection{Structs}

\definestruct{SessionID}{
	\dtypeSTRING{}
	\paramfour{}{Session Name}{str}{Value = '''' (Empty String)}
	\dtypeINT{}
	\paramfour{}{Session Owner}{int32}{Value = 0}
	\dtypeBOOL{}
	\paramfour{}{Session Is Private}{bool}{Value = True}
	\dtypeINT{}
	\paramfour{}{Session Master}{int32}{Value = 0}}

\definestruct{UserInfo}{
	\dtypeINT{}
	\paramfour{}{userType}{int32}{ProgramType Enum. Valid values range from }
	\dtypeSTRING{}
	\paramfour{}{userName}{str}{ASCII-Encoded string containing the username}
	\dtypeSTRING{}
	\paramfour{}{ipAddress}{str}{ASCII-Encoded string containing the user's IP Address}
	\dtypeSTRING{}
	\paramfour{}{hostname}{str}{ASCII-Encoded string containing the user's hostname}
	\dtypeSTRING{}
	\paramfour{}{emailAddress}{str}{ASCII-Encoded string containing the user's EMail-Address}
	\dtypeSTRING{}
	\paramfour{}{url}{str}{ASCII-Encoded string containing the user's url}}

\subsubsection{MessageType Parameters}

List of all message Types and their bodys

\begin{tcpmessage}{VRB\_CONTACT}{94}
	\paramfour{16}{Debugmode enabled}{byte}{Value = 1 $\Rightarrow$ Tokenbuffer is in Debug format}
	\dtypeINT{17}
	\paramfour{18..21}{RemoteClient ID}{int32}{ID of the remote Client}
	\structSessionID
	\structUserInfo
\end{tcpmessage}

\begin{tcpmessage}{VRB\_SET\_USERINFO}{98}
	\paramfour{16}{Debugmode enabled}{byte}{Value = 1 $\Rightarrow$ Tokenbuffer is in Debug format}
	\dtypeINT{}
	\paramfour{}{clientCount}{int}{Number of Clients in VRB's client list}
	\dtypeINT{}
	\paramfour{}{recipientPos}{int}{Position of recipient in Client list}
	
	\listparam{sessionCount}{
	\dtypeINT{}
	\paramfour{}{RemoteClient ID}{int}{}
	\structSessionID
	\structSessionID}
\end{tcpmessage}

Where the first session ID for each session is the (possibly public) session ID, and the second sessionID is the private session ID. 

\begin{tcpmessage}{VRB\_REGISTRY\_ENTRY\_CHANGED}{111}
	\dtypeINT{}
	\paramfour{}{Class ID}{int32}{ID of the class}
	\dtypeSTRING{}
	\paramfour{}{Class Name}{str}{Name of the class whose variable changed}
	\dtypeSTRING{}
	\paramfour{}{name}{str}{Name of the variable that changed}
	\dtypeDATA{}
	\paramfour{}{data}{DataHandle}{New Data of the changed Variable}
\end{tcpmessage}

\begin{tcpmessage}{VRB\_REGISTRY\_ENTRY\_DELETED}{112}
	\dtypeINT{}
	\paramfour{}{Class ID}{int32}{ID of the class}
	\dtypeSTRING{}
	\paramfour{}{Class Name}{str}{Name of the class whose variable changed}
	\dtypeSTRING{}
	\paramfour{}{name}{str}{Name of the variable that changed}
	\dtypeDATA{}
	\paramfour{}{data}{DataHandle}{New Data of the changed Variable}
\end{tcpmessage}

\begin{tcpmessage}{VRB\_REGISTRY\_SUBSCRIBE\_CLASS}{113}
	\structSessionID
	\dtypeINT{}
	\paramfour{}{classID}{int32}{ID of the class to subscribe to ? }
	\dtypeSTRING{}
	\paramfour{}{name}{str}{Name of the class to subscribe to}
\end{tcpmessage}

\begin{tcpmessage}{VRB\_REGISTRY\_SUBSCRIBE\_VARIABLE}{114}
	\structSessionID
	\dtypeINT{}
	\paramfour{}{id}{int32}{Class ID}
	\dtypeSTRING{}
	\paramfour{}{class name}{str}{Name of the variable's class to subscribe to}
	\dtypeSTRING{}
	\paramfour{}{name}{str}{Name of the variable}
\end{tcpmessage}

\begin{tcpmessage}{VRB\_REGISTRY\_CREATE\_ENTRY}{115}
	\structSessionID
	\dtypeINT{}
	\paramfour{}{ClientID}{int32}{}
	\dtypeSTRING{}
	\paramfour{}{cl}{str}{Class ?}
	\dtypeSTRING{}
	\paramfour{}{var}{str}{Variable Name}
	\dtypeDATA{}
	\paramfour{}{value}{DataHandle}{Data of the variable}
	\dtypeBOOL{}
	\paramfour{}{isStatic}{bool}{if True, variable is considered static}
\end{tcpmessage}

\begin{tcpmessage}{VRB\_REGISTRY\_SET\_VALUE}{116}
	\structSessionID
	\dtypeINT{}
	\paramfour{}{Client ID}{int32}{Local client ID}
	\dtypeSTRING{}
	\paramfour{}{cl}{str}{Class name}
	\dtypeSTRING{}
	\paramfour{}{var}{str}{Variable name}
	\dtypeDATA{}
	\paramfour{}{value}{DataHandle}{New Data of the variable}
\end{tcpmessage}

\begin{tcpmessage}{VRB\_REGISTRY\_DELETE\_ENTRY}{117}
	\structSessionID
	\dtypeINT{}
	\paramfour{}{Client ID}{int32}{Local client ID}
	\dtypeSTRING{}
	\paramfour{}{cl}{str}{Class name}
	\dtypeSTRING{}
	\paramfour{}{var}{str}{Variable name}
\end{tcpmessage}

\begin{tcpmessage}{VRB\_REGISTRY\_UNSUBSCRIBE\_CLASS}{118}
	\structSessionID
	\dtypeINT{}
	\paramfour{}{Client ID}{int32}{Local client ID}
	\dtypeSTRING{}
	\paramfour{}{cl}{str}{Class name}
\end{tcpmessage}

\begin{tcpmessage}{VRB\_REGISTRY\_UNSUBSCRIBE\_VARIABLE}{119}
		\structSessionID
	\dtypeINT{}
	\paramfour{}{Client ID}{int32}{Local client ID}
	\dtypeSTRING{}
	\paramfour{}{cl}{str}{Class name}
	\dtypeSTRING{}
	\paramfour{}{var}{str}{Variable name}
\end{tcpmessage}

\begin{tcpmessage}{VRB\_REQUEST\_NEW\_SESSION}{132}
	\structSessionID
\end{tcpmessage}

Requests the creation of a new public session. The SessionID is for the to be created public session

\begin{tcpmessage}{VRBC\_SET\_SESSION}{133}
	\dtypeINT{}
	\paramfour{}{\color{red}Sender ID}{int}{\color{red}ID of the message's sender ?}
	\structSessionID
\end{tcpmessage}

\begin{tcpmessage}{VRB\_SEND\_SESSIONS}{134}
	\paramfour{16}{Debugmode enabled}{byte}{Value = 1 $\Rightarrow$ Tokenbuffer is in Debug format}
	\dtypeINT{}
	\paramfour{}{sessionCount}{int}{Number of Sessions currently active}
	\listparam{sessionCount}{\structSessionID}
	
\end{tcpmessage}

\subsection{UDP Message}
\seclbl{protocol::udpmessage}

UDP Messages were included for the first attempt of porting covise network code to c\#.

\subsubsection{UDP Message Format}
\seclbl{protocol::udpmessage::udpmessage-format}

\begin{messagedesc}
	\messageType{type}{int}{Type of UDP Message. For a list of valid values, see \secnref{protocol::udpmessage::udpmessage-types}}
	\messageType{sender}{int}{Sender of message, sender $<$ 0 are invalid, sender 0 is the server and sender $>$ 0 are clients}
	\messageType{data}{}{Bytes containing custom data}
\end{messagedesc}

More Precisely, the header of the message consists of 2 IEEE Ints (8 Bytes), consisting of first the type, and then the sender. This is followed by the data. The entire message (consisting of both header and data) must not exceed the defined (system dependend) Write buffer size. See \lstinline|WRITE_BUFFER_SIZE| in header file of \secnref{classes::covise-connect} for the maximum total packet size. At the time of Writing, this is 393216 byte on CRAY systems, and 64000 byte on all other systems.

\subsubsection{UDP Message Types}
\seclbl{protocol::udpmessage::udpmessage-types}

\begin{messagetypes}
	\messageType{EMPTY}{0}{}
	\messageType{AVATAR_HMD_POSITION}{1}{}
	\messageType{AVATAR_CONTROLLER_POSITION}{2}{}
	\messageType{AUDIO_STREAM}{3}{}
	\messageType{MIDI_STREAM}{4}{}
\end{messagetypes}

\section{Highlevel Protocol (VRB Server)}

\subsection{Connect}

\begin{enumerate}
	\item Server sends \lstinline|0x01| (Single byte with value 1)
	\item Client sends \lstinline|0x01| (Single byte with value 1)
	\item Client sends \lstinline|COVISE_MESSAGE_VRB_CONTACT|
	\item Server responds with 	
	\begin{enumerate}
		\item Sending \lstinline|COVISE_MESSAGE_VRBC_SEND_SESSIONS| to all connected clients (Including newly connected Client)
		\item Passes on \lstinline|COVISE_MESSAGE_VRB_SET_USERINFO| to all clients (Including newly connected Client). \todo{Find difference between 'Sending' and 'Passing on'}
		\item Sends \lstinline|COVISE_MESSAGE_VRB_SET_USERINFO| to newly connected client, containing information about all other clients
		\item Send \lstinline|COVISE_MESSAGE_VRBC_SEND_SESSIONS| to newly connected client again
	\end{enumerate}
\end{enumerate}

\subsection{Create Public Session}

\begin{enumerate}
	\item Client sends \lstinline|VRB_REQUEST_NEW_SESSION| to Server
	\item Server sends \lstinline|VRB_SEND_SESSIONS| to all
	\item Server responds with \lstinline|VRBC_SET_SESSION| to client
\end{enumerate}

\section{Classes}
\seclbl{classes}

\subsection{covise\_connect}
\seclbl{classes::covise-connect}

Handles the socket connection 

\subsection{message}
\seclbl{classes::message}

Definition of standart message

\subsection{message\_types}
\seclbl{classes::message-types}

Definition of message- and sender- types 

\subsection{udp\_message\_types}
\seclbl{classes::udp-message-types}
Definition of udp message- and sender- types 

\subsection{udpMessage}
\seclbl{classes::udp-message}
Definition of udp message


\subsubsection{Trivia}

\begin{itemize}
	\item File is inconsitently named using CamelCase instead of underscores as seperators
	\item In case the Connect high level protocol is not followed (esp. if the single byte message without header is not sent first, the VRB UI becomes unresponsive without any (visible) warning, e.g. in the console)
\end{itemize}
