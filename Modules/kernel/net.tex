\renewcommand{\namespace}{kernel::net::}


\newenvironment{messagetypes}{\rowcolors{2}{evnt.light}{evnt.dark}\mbox{}\\
	\begin{longtable}{||p{7.4cm}|p{.5cm}|p{6cm}||} 
		\rowcolor{evnt.head} \hline \textbf{Name} & \textbf{ID} & \textbf{Comment} \\ \hline\hline}{\end{longtable}}

\newenvironment{messagedesc}{\rowcolors{2}{evnt.light}{evnt.dark}\mbox{}\\
	\begin{longtable}{||p{3.4cm}|p{2.5cm}|p{8cm}||}
		\rowcolor{evnt.head} \hline \textbf{Name} & \textbf{Length} & \textbf{Comment} \\ \hline\hline}{\end{longtable}}

\newcommand{\messageType}[3]{\lstinline|#1| & #2 & #3 \\
	\hline}

\newcommand{\messageTypeDep}[3]{\rowcolor{linedeprecated}\lstinline|#1| & #2 & \color{red}\textbf{!!! DEPRECATED !!! DO NOT USE !!!}\color{black}, #3 \\
	\hline}

The net folder contains basic classes related to establishing connections between client and server using sockets.

\section{Lowlevel Protocol}
\seclbl{protocol}

\subsection{Overview}

TCP Connection for regular messages, UDP Connection for UDP Messages, Optional with SSL Encryption (at least for regular messages, possibly for both). Also there has to exist a way to exchange data on the machine via shared memory.

There exist at least 2 distinct message types: Messages and UDP Messages. UDP Messages are regular messages with a stripped down header sent using the udp protocol (UDP Messages were introduced in May 2019, while regular messages exist since 1993).

The Default port used is 31000.



\subsection{Message}
\seclbl{protocol::message}

\subsubsection{Message Format}
\seclbl{protocol::message::message-format}
Defined in \secnref{classes::message}

Each Message contains the following:

\begin{messagedesc}
	\messageType{sender}{3 byte}{Sender of message, max 3 bytes}
	\messageType{send_type}{int}{Sender Type, defaults to UNDEFINED, actual size depending on Architecture and Compiler. Should be 4 bytes on most modern systems, but could be 2. For a list of valid values, see \secnref{protocol::message::sender-types}}
	\messageType{type}{int}{Message Type, defaults to EMPTY, actual size depending on Architecture and Compiler. Should be 4 bytes on most modern systems, but could be 2. For a list of valid values, see \secnref{protocol::message::message-types}}
	\messageType{data}{}{Bytes containing custom data}
\end{messagedesc}

More specifically, the Header consists of 4 IEEE ints (16 bytes):

\begin{itemize}
	\item[[0:3]] sender
	\item[[4:7]] senderType
	\item[[8:11]] messageType
	\item[[12:15]] dataLength - Length of data in bytes
\end{itemize}



\subsubsection{Message Types}
\seclbl{protocol::message::message-types}
Defined in \secnref{classes::message-types}

\begin{messagetypes}
	\messageType{EMPTY}{-1}{Used as default in message constructor, should be ignored if send to the server (see src/sys/controller/handler.cpp, handleMsg, ln 246}
	\messageType{MSG_FAILED}{0}{Generic Failed message. Used as response when an operation failed}
	\messageType{MSG_OK}{1}{Generic success message. Used as response when an operation succeeded}
	\messageType{INIT}{2}{First message sent in communication ? \textbf{Possibly Deprecated ?} Used in src/module/renderer/VRMLRenderer, check\_aws(), ln. 498}
	\messageType{FINISHED}{3}{Finish message from a rendermodule (see src/sys/controller/handler.cpp, ln. 278), sends \lstinline|COVISE_MESSAGE_UI| with parameter ''FINISHED $\backslash$ n'' in case no modules are running anymore}
	\messageType{SEND}{4}{}
	\messageType{ALLOC}{5}{}
	\messageType{UI}{6}{UI Messages. In case body starts with ''UNDO'', this is an undo-action. For more keywords, see src/sys/controller/handler.cpp, ln. 969 ff}
	\messageType{APP_CONTACT_DM}{7}{}
	\messageType{DM_CONTACT_DM}{8}{}
	\messageType{SHM_MALLOC}{9}{}
	\messageType{SHM_MALLOC_LIST}{10}{}
	\messageType{MALLOC_OK}{11}{}
	\messageType{MALLOC_LIST_OK}{12}{}
	\messageType{MALLOC_FAILED}{13}{}
	\messageType{PREPARE_CONTACT}{14}{}
	\messageType{PREPARE_CONTACT_DM}{15}{}
	\messageType{PORT}{16}{}
	\messageType{GET_SHM_KEY}{17}{}
	\messageType{NEW_OBJECT}{18}{}
	\messageType{GET_OBJECT}{19}{}
	\messageType{REGISTER_TYPE}{20}{}
	\messageType{NEW_SDS}{21}{}
	\messageType{SEND_ID}{22}{}
	\messageType{ASK_FOR_OBJECT}{23}{}
	\messageType{OBJECT_FOUND}{24}{}
	\messageType{OBJECT_NOT_FOUND}{25}{}
	\messageType{HAS_OBJECT_CHANGED}{26}{}
	\messageType{OBJECT_UPDATE}{27}{}
	\messageType{OBJECT_TRANSFER}{28}{}
	\messageType{OBJECT_FOLLOWS}{29}{}
	\messageType{OBJECT_OK}{30}{}
	\messageType{CLOSE_SOCKET}{31}{Should be ignored if send to the server (see src/sys/controller/handler.cpp, handleMsg, ln 246}
	\messageType{DESTROY_OBJECT}{32}{}
	\messageType{CTRL_DESTROY_OBJECT}{33}{}
	\messageType{QUIT}{34}{QUIT message from user interface, opencover or other sources. Handled by server, depending on parameters. Quits the current session. (see src/sys/controller/handler.cpp)}
	\messageType{START}{35}{}
	\messageType{COVISE_ERROR}{36}{Sent to all modules as \lstinline|COVISE_MESSAGE_COVISE_ERROR|. In case of error overflow, the info that there is overflow is sent instead (see src/sys/controller/handler.cpp)}
	\messageType{INOBJ}{37}{}
	\messageType{OUTOBJ}{38}{}
	\messageType{OBJECT_NO_LONGER_USED}{39}{}
	\messageType{SET_ACCESS}{40}{}
	\messageType{FINALL}{41}{Module says it has finished. Server Side resources are released, and server sends \lstinline|COVISE_MESSAGE_UI| with parameter ''FINISHED $\backslash n$'' in case no modules are running anymore (see src/sys/controller/handler.cpp)}
	\messageType{ADD_OBJECT}{42}{}
	\messageType{DELETE_OBJECT}{43}{}
	\messageType{NEW_OBJECT_VERSION}{44}{}
	\messageType{RENDER}{45}{Forwarded to all other renderers by the server (see src/sys/controller/handler.cpp)}
	\messageType{WAIT_CONTACT}{46}{}
	\messageType{PARINFO}{47}{send message to all userinterfaces (see src/sys/controller/handler.cpp)}
	\messageType{MAKE_DATA_CONNECTION}{48}{}
	\messageType{COMPLETE_DATA_CONNECTION}{49}{}
	\messageType{SHM_FREE}{50}{}
	\messageType{GET_TRANSFER_PORT}{51}{}
	\messageType{TRANSFER_PORT}{52}{}
	\messageType{CONNECT_TRANSFERMANAGER}{53}{}
	\messageType{STDINOUT_EMPTY}{54}{}
	\messageType{WARNING}{55}{Messages are relayed as Error to all renderers. Data will be modified by prefix ''WARNING'' and suffix ''$\backslash n$'' (see src/sys/controller/handler.cpp)}
	\messageType{INFO}{56}{Messages are relayed as Error to all renderers. Data will be modified by prefix ''INFO'' and suffix ''$\backslash n$'' (see src/sys/controller/handler.cpp)}
	\messageType{REPLACE_OBJECT}{57}{}
	\messageType{PLOT}{58}{Forwarded to all other renderers by the server (see src/sys/controller/handler.cpp)}
	\messageType{GET_LIST_OF_INTERFACES}{59}{}
	\messageType{USR1}{60}{}
	\messageType{USR2}{61}{}
	\messageType{USR3}{62}{}
	\messageType{USR4}{63}{}
	\messageType{NEW_OBJECT_OK}{64}{}
	\messageType{NEW_OBJECT_FAILED}{65}{}
	\messageType{NEW_OBJECT_SHM_MALLOC_LIST}{66}{}
	\messageType{REQ_UI}{67}{Relayed to all Userinterface types (see src/sys/controller/handler.cpp)}
	\messageType{NEW_PART_ADDED}{68}{}
	\messageType{SENDING_NEW_PART}{69}{}
	\messageType{FINPART}{70}{}
	\messageType{NEW_PART_AVAILABLE}{71}{}
	\messageType{OBJECT_ON_HOSTS}{72}{}
	\messageType{OBJECT_FOLLOWS_CONT}{73}{}
	\messageType{CRB_EXEC}{74}{}
	\messageType{COVISE_STOP_PIPELINE}{75}{Sets the status of the module with the name from the parameters to stopping (see src/sys/controller/handler.cpp)}
	\messageType{PREPARE_CONTACT_MODULE}{76}{}
	\messageType{MODULE_CONTACT_MODULE}{77}{}
	\messageType{SEND_APPL_PROCID}{78}{}
	\messageType{INTERFACE_LIST}{79}{}
	\messageType{MODULE_LIST}{80}{}
	\messageType{HOSTID}{81}{}
	\messageType{MODULE_STARTED}{82}{}
	\messageType{GET_USER}{83}{}
	\messageType{SOCKET_CLOSED}{84}{Should be ignored if send to the server (see src/sys/controller/handler.cpp, handleMsg, ln 246 (see src/sys/controller/handler.cpp)}
	\messageType{NEW_COVISED}{85}{}
	\messageType{USER_LIST}{86}{}
	\messageType{STARTUP_INFO}{87}{}
	\messageType{CO_MODULE}{88}{}
	\messageType{WRITE_SCRIPT}{89}{}
	\messageType{CRB}{90}{}
	\messageType{GENERIC}{91}{Evaluated for Keywords, and handled accordingly. Message Body is parsed (see src/sys/controller/handler.cpp)}
	\messageType{RENDER_MODULE}{92}{Forwarded to all other renderers by the server (see src/sys/controller/handler.cpp)}
	\messageType{FEEDBACK}{93}{Messages from Renderer sent to a module. Message is parsed and sent to the specified module (see src/sys/controller/handler.cpp)}
	\messageType{VRB_CONTACT}{94}{Initialise connection from client}
	\messageType{VRB_CONNECT_TO_COVISE}{95}{}
	\messageType{END_IMM_CB}{96}{}
	\messageType{NEW_DESK}{97}{}
	\messageType{VRB_SET_USERINFO}{98}{}
	\messageType{VRB_GET_ID}{99}{}
	\messageType{VRB_SET_GROUP}{100}{}
	\messageType{VRB_QUIT}{101}{}
	\messageType{VRB_SET_MASTER}{102}{}
	\messageType{VRB_GUI}{103}{}
	\messageType{VRB_CLOSE_VRB_CONNECTION}{104}{}
	\messageType{VRB_REQUEST_FILE}{105}{}
	\messageType{VRB_SEND_FILE}{106}{}
	\messageTypeDep{VRB_CURRENT_FILE}{107}{use \lstinline|COVISE_MESSAGE_VRB_REQUEST _FILE| and sharedState \lstinline|coVRFileManager_filePaths| instead. Should write to standard error, then abort execution. (See kernel/vrb/server/VrbMessageHandler.cpp)}
	\messageType{CRB_QUIT}{108}{}
	\messageType{REMOVED_HOST}{109}{}
	\messageType{START_COVER_SLAVE}{110}{}
	\messageType{VRB_REGISTRY_ENTRY_CHANGED}{111}{}
	\messageType{VRB_REGISTRY_ENTRY_DELETED}{112}{}
	\messageType{VRB_REGISTRY_SUBSCRIBE_CLASS}{113}{}
	\messageType{VRB_REGISTRY_SUBSCRIBE_VARIABLE}{114}{}
	\messageType{VRB_REGISTRY_CREATE_ENTRY}{115}{}
	\messageType{VRB_REGISTRY_SET_VALUE}{116}{}
	\messageType{VRB_REGISTRY_DELETE_ENTRY}{117}{}
	\messageType{VRB_REGISTRY_UNSUBSCRIBE_CLASS}{118}{}
	\messageType{VRB_REGISTRY_UNSUBSCRIBE_VARIABLE}{119}{}
	\messageType{SYNCHRONIZED_ACTION}{120}{}
	\messageType{ACCESSGRID_DAEMON}{121}{}
	\messageType{TABLET_UI}{122}{}
	\messageType{QUERY_DATA_PATH}{123}{}
	\messageType{SEND_DATA_PATH}{124}{}
	\messageType{VRB_FB_RQ}{125}{}
	\messageType{VRB_FB_SET}{126}{}
	\messageType{VRB_FB_REMREQ}{127}{}
	\messageType{UPDATE_LOADED_MAPNAME}{128}{Relayed to all Userinterface types, MAKRO (''UPDATE\_LOADED\_MAPNAME'') executed (see src/sys/controller/handler.cpp)}
	\messageType{SSLDAEMON}{129}{}
	\messageType{VISENSO_UI}{130}{}
	\messageType{PARAMDESC}{131}{}
	\messageType{VRB_REQUEST_NEW_SESSION}{132}{}
	\messageType{VRBC_SET_SESSION}{133}{}
	\messageType{VRBC_SEND_SESSIONS}{134}{}
	\messageType{VRBC_CHANGE_SESSION}{135}{}
	\messageType{VRBC_UNOBSERVE_SESSION}{136}{}
	\messageType{VRB_SAVE_SESSION}{137}{}
	\messageType{VRB_LOAD_SESSION}{138}{}
	\messageType{VRB_MESSAGE}{139}{}
	\messageType{VRB_PERMIT_LAUNCH}{140}{}
	\messageType{BROADCAST_TO_PROGRAM}{141}{}
	\messageType{NEW_UI}{142}{Processed and handled by server. Seems to request current collaborative state or list of partners, depending on parameters (see src/sys/controller/handler.cpp)}
	\messageType{PROXY}{143}{}
	\messageType{SOUND}{144}{}
	\messageType{LAST_DUMMY_MESSAGE}{145}{}
\end{messagetypes}

\subsubsection{Sender Types}
\seclbl{protocol::message::sender-types}

Defined in \secnref{classes::message-types}

\begin{messagetypes}
	\messageType{UNDEFINED}{0}{Used as default value in message constructor}
	\messageType{CONTROLLER}{1}{}
	\messageType{CRB}{2}{}
	\messageType{USERINTERFACE}{3}{}
	\messageType{RENDERER}{4}{}
	\messageType{APPLICATIONMODULE}{5}{}
	\messageType{TRANSFERMANAGER}{6}{}
	\messageType{SIMPLEPROCESS}{7}{}
	\messageType{SIMPLECONTROLLER}{8}{}
	\messageType{STDINOUT}{9}{}
	\messageType{COVISED}{10}{}
	\messageType{VRB}{11}{}
	\messageType{SENDER_SOUND}{12}{}
	\messageType{ANY}{}{}
\end{messagetypes}

\subsection{UDP Message}
\seclbl{protocol::udpmessage}

UDP Messages were included for the first attempt of porting covise network code to c\#.

\subsubsection{UDP Message Format}
\seclbl{protocol::udpmessage::udpmessage-format}

\begin{messagedesc}
	\messageType{type}{int}{Type of UDP Message. For a list of valid values, see \secnref{protocol::udpmessage::udpmessage-types}}
	\messageType{sender}{int}{Sender of message, sender $<$ 0 are invalid, sender 0 is the server and sender $>$ 0 are clients}
	\messageType{data}{}{Bytes containing custom data}
\end{messagedesc}

More Precisely, the header of the message consists of 2 IEEE Ints (8 Bytes), consisting of first the type, and then the sender. This is followed by the data. The entire message (consisting of both header and data) must not exceed the defined (system dependend) Write buffer size. See \lstinline|WRITE_BUFFER_SIZE| in header file of \secnref{classes::covise-connect} for the maximum total packet size. At the time of Writing, this is 393216 byte on CRAY systems, and 64000 byte on all other systems.

\subsubsection{UDP Message Types}
\seclbl{protocol::udpmessage::udpmessage-types}

\begin{messagetypes}
	\messageType{EMPTY}{0}{}
	\messageType{AVATAR_HMD_POSITION}{1}{}
	\messageType{AVATAR_CONTROLLER_POSITION}{2}{}
	\messageType{AUDIO_STREAM}{3}{}
	\messageType{MIDI_STREAM}{4}{}
\end{messagetypes}

\section{Highlevel Protocol}

\subsection{Connect}

\begin{enumerate}
	\item Send \lstinline|COVISE_MESSAGE_VRB_CONTACT|	
\end{enumerate}


\section{Classes}
\seclbl{classes}

\subsection{covise\_connect}
\seclbl{classes::covise-connect}

Handles the socket connection 

\subsection{message}
\seclbl{classes::message}

Definition of standart message

\subsection{message\_types}
\seclbl{classes::message-types}

Definition of message- and sender- types 

\subsection{udp\_message\_types}
\seclbl{classes::udp-message}
Definition of udp message- and sender- types 

\subsection{udpMessage}
\seclbl{classes::udp-message}
Definition of udp message


\subsubsection{Trivia}

\begin{itemize}
	\item File is inconsitently named using CamelCase instead of underscores as seperators
\end{itemize}

\subsection{udp\_message\_types}
\seclbl{classes::udp-message-types}