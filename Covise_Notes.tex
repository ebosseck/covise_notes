\documentclass[12pt,pdftex,a4paper]{scrbook}
\usepackage[utf8]{inputenc}
\usepackage[english]{babel}
\usepackage{amsmath}
\usepackage{amssymb}
\usepackage[pdftex]{graphicx}
\usepackage{amsfonts}
\usepackage{listings}
\usepackage[table]{xcolor}
\usepackage{tcolorbox}

\usepackage{hyperref}
\usepackage{fancyhdr}

\usepackage{hyperref}
\usepackage{colortbl}
\usepackage{multirow}

\usepackage{longtable}

% !TeX root = ../Covise_Notes.tex

\definecolor{HEADDER}{rgb}{0.9, 0.9, 1}
\definecolor{ODD}{gray}{1.0} 
\definecolor{EVEN}{gray}{0.9}

\definecolor{hint.info}      {rgb} {0, 0, 1}
\definecolor{hint.example}   {rgb} {0, 1, 0} 
\definecolor{hint.important} {rgb} {1, 0, 0}
\definecolor{marking}{rgb}{0.6, 0, 0}

\definecolor{prim.head} {rgb} {.6, .6, .6}
\definecolor{prim.light} {rgb} {.8, .8, .8}
\definecolor{prim.dark} {rgb} {1, 1, 1}

\definecolor{send.head} {rgb} {.5, 1, .5}
\definecolor{send.light} {rgb} {.6, 1, .6}
\definecolor{send.dark} {rgb} {.8, 1, .8}

\definecolor{recv.head} {rgb} {1, .5, .5}
\definecolor{recv.light} {rgb} {1, .6, .6}
\definecolor{recv.dark} {rgb} {1, .8, .8}

\definecolor{evnt.head} {rgb} {.5, .5, 1} %{1, 1, .5}
\definecolor{evnt.light} {rgb} {.6, .6, 1} %{1, 1, .6}
\definecolor{evnt.dark} {rgb} {.8, .8, 1} %{1, 1, .8}

\definecolor{linedeprecated}{rgb}{1, 1, .6}

\definecolor{typedef.head} {rgb} {.5, .5, 1}
\definecolor{typedef.light} {rgb} {.6, .6, 1}
\definecolor{typedef.dark} {rgb} {.8, .8, 1}

\newenvironment{info}{\begin{tcolorbox}[colback=hint.info!2,colframe=hint.info!70!black,title=\textsc{Info}]}{\end{tcolorbox}}
\newenvironment{infocont}{\begin{tcolorbox}[colback=hint.info!2,colframe=hint.info!70!black,title=\textsc{Info (Cont.)}]}{\end{tcolorbox}}
\newenvironment{wichtig}{\begin{tcolorbox}[colback=hint.important!2,colframe=hint.important!70!black,title=\textsc{Important}]}{\end{tcolorbox}}
\newenvironment{beispiel}{\begin{tcolorbox}[colback=hint.example!2,colframe=hint.example!70!black,title=\textsc{Example}]}{\end{tcolorbox}}

\newenvironment{beispielcont}{\begin{tcolorbox}[colback=hint.example!2,colframe=hint.example!70!black,title=\textsc{Example (Cont.)}]}{\end{tcolorbox}}

\newenvironment{additional}{}{}

\newcommand{\therequest}{}

\newenvironment{request}[1]{\renewcommand{\therequest}{#1}\subsubsection{#1}}{}

\newenvironment{send}{\rowcolors{2}{send.light}{send.dark}\paragraph*{Request}\mbox{}\\
	\begin{longtable}{||p{4cm}|p{3cm}|p{7cm}||}
		\rowcolor{send.head} \hline \textbf{Name} & \textbf{Type}& \textbf{Description} \\
		\hline\hline
		\param{request}{str}{Request Type (''\texttt{\therequest}'')}
		\param{messageid}{str}{Message ID, this parameter is mirrored by the server}
	}{\end{longtable}}

\newenvironment{recv}{\rowcolors{2}{recv.light}{recv.dark}\paragraph*{Response}\mbox{}\\
	\begin{longtable}{||p{4cm}|p{3cm}|p{7cm}||}
		\rowcolor{recv.head} \hline \textbf{Name} & \textbf{Type}& \textbf{Description} \\ \hline\hline
		\param{messageid}{str}{Message ID, this parameter is mirrored by the server}}{\end{longtable}}

\newenvironment{evnt}{\rowcolors{2}{evnt.light}{evnt.dark}\paragraph*{Message}\mbox{}\\
	\begin{longtable}{||p{4cm}|p{3cm}|p{7cm}||}
		\rowcolor{evnt.head} \hline \textbf{Name} & \textbf{Type}& \textbf{Description} \\ \hline\hline
		\param{event}{str}{Name of this event}
		\param{source}{str}{Source of this event}}{\end{longtable}}

\newenvironment{type}[1]{\renewcommand{\therequest}{#1}\subsection{#1}}{}

\newenvironment{typedef}{\rowcolors{2}{typedef.light}{typedef.dark}
	\begin{longtable}{||p{4cm}|p{3cm}|p{7cm}||}
		\rowcolor{typedef.head} \hline \textbf{Name} & \textbf{Type}& \textbf{Description} \\ \hline\hline}{\end{longtable}}

\newcommand{\param}[3]{\texttt{#1} & \textsl{#2} & #3 \\\hline}
\newcommand{\primitive}[2]{\textsl{#1} & #2 \\\hline}

\newcommand{\triggers}[1]{\textbf{Triggers Event:} #1}



\newcommand{\keyword}[1]{\textbf{\color{marking}#1}}
\newcommand{\figref}[1]{Figure~\ref{#1}}
\newcommand{\todo}[1]{\textbf{\color{red}TODO: #1}}

\newcommand{\sitem}[1]{\begin{itemize}
		\item #1
\end{itemize}}

\newcommand{\plang}[1]{\textit{#1}}

\newcommand*{\pdfimg}[2]{
	{\centering
		\includegraphics[width=#1\textwidth]{fig/fig_#2.pdf}}
}

\newcommand*{\pdffig}[3]{\begin{figure}[h!]
		\centering
		\includegraphics[width=#1\textwidth]{fig/fig_#2.pdf}
		\caption{%
			#3
		}
		\label{fig:fig_#2}
\end{figure}}

\newcommand{\sidebyside}[2]{\begin{minipage}{0.45\textwidth}
		#1
	\end{minipage} \hspace{0.1\textwidth}
	\begin{minipage}{0.45\textwidth}
		#2
\end{minipage}}

\newcommand*{\pdffigs}[4]{\begin{figure}[h!]
		\centering
		\includegraphics[width=#1\textwidth]{fig/fig_#2.pdf}
		\includegraphics[width=#1\textwidth]{fig/fig_#3.pdf}
		\caption{%
			#4
		}
		\label{fig:fig_#2}
\end{figure}}

\setcounter{MaxMatrixCols}{20}
%\rowcolors{2}{EVEN}{ODD}
\newcommand{\mathbox}[1]{\text{\fbox{$\displaystyle#1$}}}


\newcommand{\filenotfoundmsg}{File not included. Please contact original author.}
\newcommand{\inputExists}[1]{\IfFileExists{#1}{\input{#1}}{\filenotfoundmsg}}

%#################################
%#####      LSTLISTINGS      #####
%#################################


\lstset{ 
	backgroundcolor=\color{white},   % choose the background color; you must add \usepackage{color} or \usepackage{xcolor}; should come as last argument
	basicstyle=\footnotesize,        % the size of the fonts that are used for the code
	breakatwhitespace=false,         % sets if automatic breaks should only happen at whitespace
	breaklines=true,                 % sets automatic line breaking
	captionpos=b,                    % sets the caption-position to bottom
	commentstyle=\color{green!60!black},      % comment style
	deletekeywords={...},            % if you want to delete keywords from the given language
	escapeinside={\%*}{*)},          % if you want to add LaTeX within your code
	extendedchars=true,              % lets you use non-ASCII characters; for 8-bits encodings only, does not work with UTF-8
	frame=all,	                 % adds a frame around the code
	keepspaces=true,                 % keeps spaces in text, useful for keeping indentation of code (possibly needs columns=flexible)
	keywordstyle=\color{violet},     % keyword style
	language=C,                 	 % the language of the code
	morekeywords={*,..., fi, do, od, return, then},            % if you want to add more keywords to the set
	numbers=left,                    % where to put the line-numbers; possible values are (none, left, right)
	numbersep=5pt,                   % how far the line-numbers are from the code
	numberstyle=\tiny\color{gray},   % the style that is used for the line-numbers
	rulecolor=\color{black!50},         % if not set, the frame-color may be changed on line-breaks within not-black text (e.g. comments (green here))
	showspaces=false,                % show spaces everywhere adding particular underscores; it overrides 'showstringspaces'
	showstringspaces=false,          % underline spaces within strings only
	showtabs=false,                  % show tabs within strings adding particular underscores
	stepnumber=1,                    % the step between two line-numbers. If it's 1, each line will be numbered
	stringstyle=\color{blue},        % string literal style
	tabsize=2,	                     % sets default tabsize to 2 spaces
	title=\lstname                   % show the filename of files included with \lstinputlisting; also try caption instead of title
}


\lstdefinelanguage
[x64]{Assembler}     % add a "x64" dialect of Assembler
[x86masm]{Assembler} % based on the "x86masm" dialect
% with these extra keywords:
{morekeywords={CDQE,CQO,CMPSQ,CMPXCHG16B,JRCXZ,LODSQ,MOVSXD, %
		POPFQ,PUSHFQ,SCASQ,STOSQ,IRETQ,RDTSCP,SWAPGS, %
		rax,rdx,rcx,rbx,rsi,rdi,rsp,rbp, %
		r8,r8d,r8w,r8b,r9,r9d,r9w,r9b, %
		r10,r10d,r10w,r10b,r11,r11d,r11w,r11b, %
		r12,r12d,r12w,r12b,r13,r13d,r13w,r13b, %
		r14,r14d,r14w,r14b,r15,r15d,r15w,r15b}} % etc.

\lstdefinelanguage{JavaScript}{
	keywords={break, case, catch, continue, debugger, default, delete, do, else, finally, for, function, if, in, instanceof, new, return, switch, this, throw, try, typeof, var, void, while, with},
	morecomment=[l]{//},
	morecomment=[s]{/*}{*/},
	morestring=[b]',
	morestring=[b]",
	sensitive=true
}

\lstdefinelanguage{Scheme}{
	keywords={define, lambda},
	morekeywords=[1]{define, define-syntax, define-macro, lambda, define-stream, stream-lambda},
	morekeywords=[2]{begin, call-with-current-continuation, call/cc,
		call-with-input-file, call-with-output-file, case, cond,
		do, else, for-each, if,
		let*, let, let-syntax, letrec, letrec-syntax,
		let-values, let*-values,
		and, or, not, delay, force,
		quasiquote, quote, unquote, unquote-splicing,
		map, fold, syntax, syntax-rules, eval, environment, query, car, cdr, cons, set!, set-car!, set-cdr!},
	morekeywords=[3]{import, export},
	morecomment=[l]{;},
	morecomment=[s]{\#|}{|\#},
	alsodigit=!\$\%&*+-./:<=>?@^_~,
	literate=*{`}{{`}}{1},
	sensitive=true
}

\setcounter{tocdepth}{6}
\setcounter{secnumdepth}{6}

\newcommand{\namespace}{}
\newcommand{\resolveNS}[1]{\namespace#1}
\newcommand{\seclbl}[1]{\label{\resolveNS{#1}}}
\newcommand{\secnref}[1]{\nameref{\resolveNS{#1}}}

\newcommand{\shand}[1]{\nameref{intern::shorthands::#1}}
\newcommand{\shandentry}[2]{\label{intern::shorthands::#1} #1 & #2 \\ \hline}


\begin{document}
	\thispagestyle{empty}
	\title{Covise}
	\subtitle{Notes}
	\author{}
	\maketitle
	
	\newpage
	\tableofcontents

	\part{Kernel}
	
	\chapter{net}
	
	\renewcommand{\namespace}{kernel::net::}


\newenvironment{messagetypes}{\rowcolors{2}{evnt.light}{evnt.dark}\mbox{}\\
	\begin{longtable}{||p{7.4cm}|p{.5cm}|p{6cm}||} 
		\rowcolor{evnt.head} \hline \textbf{Name} & \textbf{ID} & \textbf{Comment} \\ \hline\hline}{\end{longtable}}

\newenvironment{messagedesc}{\rowcolors{2}{evnt.light}{evnt.dark}\mbox{}\\
	\begin{longtable}{||p{3.4cm}|p{2.5cm}|p{8cm}||}
		\rowcolor{evnt.head} \hline \textbf{Name} & \textbf{Length} & \textbf{Comment} \\ \hline\hline}{\end{longtable}}

\newcommand{\messageType}[3]{\lstinline|#1| & #2 & #3 \\
	\hline}

The net folder contains basic classes related to establishing connections between client and server using sockets.

\section{Protocol}
\seclbl{protocol}

\subsection{Overview}

TCP Connection for regular messages, UDP Connection for UDP Messages, Optional with SSL Encryption (at least for regular messages, possibly for both). Also there has to exist a way to exchange data on the machine via shared memory.

There exist at least 2 distinct message types: Messages and UDP Messages. UDP Messages are regular messages with a stripped down header sent using the udp protocol (UDP Messages were introduced in May 2019, while regular messages exist since 1993).

The Default port used is 31000.



\subsection{Message}
\seclbl{protocol::message}

\subsubsection{Message Format}
\seclbl{protocol::message::message-format}
Defined in \secnref{classes::message}

Each Message contains the following:

\begin{messagedesc}
	\messageType{sender}{3 byte}{Sender of message, max 3 bytes}
	\messageType{send_type}{int}{Sender Type, defaults to UNDEFINED, actual size depending on Architecture and Compiler. Should be 4 bytes on most modern systems, but could be 2. For a list of valid values, see \secnref{protocol::message::sender-types}}
	\messageType{type}{int}{Message Type, defaults to EMPTY, actual size depending on Architecture and Compiler. Should be 4 bytes on most modern systems, but could be 2. For a list of valid values, see \secnref{protocol::message::message-types}}
	\messageType{data}{}{Bytes containing custom data}
\end{messagedesc}

More specifically, the Header consists of 4 IEEE ints (16 bytes):

\begin{itemize}
	\item[[0:3]] sender
	\item[[4:7]] senderType
	\item[[8:11]] messageType
	\item[[12:15]] dataLength - Length of data in bytes
\end{itemize}



\subsubsection{Message Types}
\seclbl{protocol::message::message-types}
Defined in \secnref{classes::message-types}

\begin{messagetypes}
	\messageType{EMPTY}{-1}{}
	\messageType{MSG_FAILED}{0}{}
	\messageType{MSG_OK}{1}{}
	\messageType{INIT}{2}{}
	\messageType{FINISHED}{3}{}
	\messageType{SEND}{4}{}
	\messageType{ALLOC}{5}{}
	\messageType{UI}{6}{}
	\messageType{APP_CONTACT_DM}{7}{}
	\messageType{DM_CONTACT_DM}{8}{}
	\messageType{SHM_MALLOC}{9}{}
	\messageType{SHM_MALLOC_LIST}{10}{}
	\messageType{MALLOC_OK}{11}{}
	\messageType{MALLOC_LIST_OK}{12}{}
	\messageType{MALLOC_FAILED}{13}{}
	\messageType{PREPARE_CONTACT}{14}{}
	\messageType{PREPARE_CONTACT_DM}{15}{}
	\messageType{PORT}{16}{}
	\messageType{GET_SHM_KEY}{17}{}
	\messageType{NEW_OBJECT}{18}{}
	\messageType{GET_OBJECT}{19}{}
	\messageType{REGISTER_TYPE}{20}{}
	\messageType{NEW_SDS}{21}{}
	\messageType{SEND_ID}{22}{}
	\messageType{ASK_FOR_OBJECT}{23}{}
	\messageType{OBJECT_FOUND}{24}{}
	\messageType{OBJECT_NOT_FOUND}{25}{}
	\messageType{HAS_OBJECT_CHANGED}{26}{}
	\messageType{OBJECT_UPDATE}{27}{}
	\messageType{OBJECT_TRANSFER}{28}{}
	\messageType{OBJECT_FOLLOWS}{29}{}
	\messageType{OBJECT_OK}{30}{}
	\messageType{CLOSE_SOCKET}{31}{}
	\messageType{DESTROY_OBJECT}{32}{}
	\messageType{CTRL_DESTROY_OBJECT}{33}{}
	\messageType{QUIT}{34}{}
	\messageType{START}{35}{}
	\messageType{COVISE_ERROR}{36}{}
	\messageType{INOBJ}{37}{}
	\messageType{OUTOBJ}{38}{}
	\messageType{OBJECT_NO_LONGER_USED}{39}{}
	\messageType{SET_ACCESS}{40}{}
	\messageType{FINALL}{41}{}
	\messageType{ADD_OBJECT}{42}{}
	\messageType{DELETE_OBJECT}{43}{}
	\messageType{NEW_OBJECT_VERSION}{44}{}
	\messageType{RENDER}{45}{}
	\messageType{WAIT_CONTACT}{46}{}
	\messageType{PARINFO}{47}{}
	\messageType{MAKE_DATA_CONNECTION}{48}{}
	\messageType{COMPLETE_DATA_CONNECTION}{49}{}
	\messageType{SHM_FREE}{50}{}
	\messageType{GET_TRANSFER_PORT}{51}{}
	\messageType{TRANSFER_PORT}{52}{}
	\messageType{CONNECT_TRANSFERMANAGER}{53}{}
	\messageType{STDINOUT_EMPTY}{54}{}
	\messageType{WARNING}{55}{}
	\messageType{INFO}{56}{}
	\messageType{REPLACE_OBJECT}{57}{}
	\messageType{PLOT}{58}{}
	\messageType{GET_LIST_OF_INTERFACES}{59}{}
	\messageType{USR1}{60}{}
	\messageType{USR2}{61}{}
	\messageType{USR3}{62}{}
	\messageType{USR4}{63}{}
	\messageType{NEW_OBJECT_OK}{64}{}
	\messageType{NEW_OBJECT_FAILED}{65}{}
	\messageType{NEW_OBJECT_SHM_MALLOC_LIST}{66}{}
	\messageType{REQ_UI}{67}{}
	\messageType{NEW_PART_ADDED}{68}{}
	\messageType{SENDING_NEW_PART}{69}{}
	\messageType{FINPART}{70}{}
	\messageType{NEW_PART_AVAILABLE}{71}{}
	\messageType{OBJECT_ON_HOSTS}{72}{}
	\messageType{OBJECT_FOLLOWS_CONT}{73}{}
	\messageType{CRB_EXEC}{74}{}
	\messageType{COVISE_STOP_PIPELINE}{75}{}
	\messageType{PREPARE_CONTACT_MODULE}{76}{}
	\messageType{MODULE_CONTACT_MODULE}{77}{}
	\messageType{SEND_APPL_PROCID}{78}{}
	\messageType{INTERFACE_LIST}{79}{}
	\messageType{MODULE_LIST}{80}{}
	\messageType{HOSTID}{81}{}
	\messageType{MODULE_STARTED}{82}{}
	\messageType{GET_USER}{83}{}
	\messageType{SOCKET_CLOSED}{84}{}
	\messageType{NEW_COVISED}{85}{}
	\messageType{USER_LIST}{86}{}
	\messageType{STARTUP_INFO}{87}{}
	\messageType{CO_MODULE}{88}{}
	\messageType{WRITE_SCRIPT}{89}{}
	\messageType{CRB}{90}{}
	\messageType{GENERIC}{91}{}
	\messageType{RENDER_MODULE}{92}{}
	\messageType{FEEDBACK}{93}{}
	\messageType{VRB_CONTACT}{94}{Anmeldung Client}
	\messageType{VRB_CONNECT_TO_COVISE}{95}{}
	\messageType{END_IMM_CB}{96}{}
	\messageType{NEW_DESK}{97}{}
	\messageType{VRB_SET_USERINFO}{98}{}
	\messageType{VRB_GET_ID}{99}{}
	\messageType{VRB_SET_GROUP}{100}{}
	\messageType{VRB_QUIT}{101}{}
	\messageType{VRB_SET_MASTER}{102}{}
	\messageType{VRB_GUI}{103}{}
	\messageType{VRB_CLOSE_VRB_CONNECTION}{104}{}
	\messageType{VRB_REQUEST_FILE}{105}{}
	\messageType{VRB_SEND_FILE}{106}{}
	\messageType{VRB_CURRENT_FILE}{107}{}
	\messageType{CRB_QUIT}{108}{}
	\messageType{REMOVED_HOST}{109}{}
	\messageType{START_COVER_SLAVE}{110}{}
	\messageType{VRB_REGISTRY_ENTRY_CHANGED}{111}{}
	\messageType{VRB_REGISTRY_ENTRY_DELETED}{112}{}
	\messageType{VRB_REGISTRY_SUBSCRIBE_CLASS}{113}{}
	\messageType{VRB_REGISTRY_SUBSCRIBE_VARIABLE}{114}{}
	\messageType{VRB_REGISTRY_CREATE_ENTRY}{115}{}
	\messageType{VRB_REGISTRY_SET_VALUE}{116}{}
	\messageType{VRB_REGISTRY_DELETE_ENTRY}{117}{}
	\messageType{VRB_REGISTRY_UNSUBSCRIBE_CLASS}{118}{}
	\messageType{VRB_REGISTRY_UNSUBSCRIBE_VARIABLE}{119}{}
	\messageType{SYNCHRONIZED_ACTION}{120}{}
	\messageType{ACCESSGRID_DAEMON}{121}{}
	\messageType{TABLET_UI}{122}{}
	\messageType{QUERY_DATA_PATH}{123}{}
	\messageType{SEND_DATA_PATH}{124}{}
	\messageType{VRB_FB_RQ}{125}{}
	\messageType{VRB_FB_SET}{126}{}
	\messageType{VRB_FB_REMREQ}{127}{}
	\messageType{UPDATE_LOADED_MAPNAME}{128}{}
	\messageType{SSLDAEMON}{129}{}
	\messageType{VISENSO_UI}{130}{}
	\messageType{PARAMDESC}{131}{}
	\messageType{VRB_REQUEST_NEW_SESSION}{132}{}
	\messageType{VRBC_SET_SESSION}{133}{}
	\messageType{VRBC_SEND_SESSIONS}{134}{}
	\messageType{VRBC_CHANGE_SESSION}{135}{}
	\messageType{VRBC_UNOBSERVE_SESSION}{136}{}
	\messageType{VRB_SAVE_SESSION}{137}{}
	\messageType{VRB_LOAD_SESSION}{138}{}
	\messageType{VRB_MESSAGE}{139}{}
	\messageType{VRB_PERMIT_LAUNCH}{140}{}
	\messageType{BROADCAST_TO_PROGRAM}{141}{}
	\messageType{NEW_UI}{142}{}
	\messageType{PROXY}{143}{}
	\messageType{SOUND}{144}{}
	\messageType{LAST_DUMMY_MESSAGE}{145}{}
\end{messagetypes}

\subsubsection{Sender Types}
\seclbl{protocol::message::sender-types}

Defined in \secnref{classes::message-types}

\begin{messagetypes}
	\messageType{UNDEFINED}{0}{}
	\messageType{CONTROLLER}{1}{}
	\messageType{CRB}{2}{}
	\messageType{USERINTERFACE}{3}{}
	\messageType{RENDERER}{4}{}
	\messageType{APPLICATIONMODULE}{5}{}
	\messageType{TRANSFERMANAGER}{6}{}
	\messageType{SIMPLEPROCESS}{7}{}
	\messageType{SIMPLECONTROLLER}{8}{}
	\messageType{STDINOUT}{9}{}
	\messageType{COVISED}{10}{}
	\messageType{VRB}{11}{}
	\messageType{SENDER_SOUND}{12}{}
	\messageType{ANY}{}{}
\end{messagetypes}

\subsection{UDP Message}
\seclbl{protocol::udpmessage}

\subsubsection{UDP Message Format}
\seclbl{protocol::udpmessage::udpmessage-format}

\begin{messagedesc}
	\messageType{type}{int}{Type of UDP Message. For a list of valid values, see \secnref{protocol::udpmessage::udpmessage-types}}
	\messageType{sender}{int}{Sender of message, sender $<$ 0 are invalid, sender 0 is the server and sender $>$ 0 are clients}
	\messageType{data}{}{Bytes containing custom data}
\end{messagedesc}

More Precisely, the header of the message consists of 2 IEEE Ints (8 Bytes), consisting of first the type, and then the sender. This is followed by the data. The entire message (consisting of both header and data) must not exceed the defined (system dependend) Write buffer size. See \lstinline|WRITE_BUFFER_SIZE| in header file of \secnref{classes::covise-connect} for the maximum total packet size. At the time of Writing, this is 393216 byte on CRAY systems, and 64000 byte on all other systems.

\subsubsection{UDP Message Types}
\seclbl{protocol::udpmessage::udpmessage-types}

\begin{messagetypes}
	\messageType{EMPTY}{0}{}
	\messageType{AVATAR_HMD_POSITION}{1}{}
	\messageType{AVATAR_CONTROLLER_POSITION}{2}{}
	\messageType{AUDIO_STREAM}{3}{}
	\messageType{MIDI_STREAM}{4}{}
\end{messagetypes}

\section{Classes}
\seclbl{classes}

\subsection{covise\_connect}
\seclbl{classes::covise-connect}

Handles the socket connection 

\subsection{message}
\seclbl{classes::message}

Definition of standart message

\subsection{message\_types}
\seclbl{classes::message-types}

Definition of message- and sender- types 

\subsection{udp\_message\_types}
\seclbl{classes::udp-message}
Definition of udp message- and sender- types 

\subsection{udpMessage}
\seclbl{classes::udp-message}
Definition of udp message


\subsubsection{Trivia}

\begin{itemize}
	\item File is inconsitently named using CamelCase instead of underscores as seperators
\end{itemize}

\subsection{udp\_message\_types}
\seclbl{classes::udp-message-types}
	
	\chapter{vrb}

	% !TeX root = ../../Covise_Notes.tex

\renewcommand{\namespace}{kernel::vrb::}

\section{Classes}

\subsection{server/VrbMessageHandler}

Handles messages sent to Vrb Server

\subsubsection{Trivia}
\begin{itemize}
	\item handleFileBrouwserRequest method name contains typo
\end{itemize}

	\part{Remarks}

	\chapter{Shorthands}
	
	\begin{tabular}{||p{4cm}|p{10cm}||}
		\hline
		\textbf{Shorthand} & \textbf{Description}\\
		\hline
		\hline
		\shandentry{aws}{?}
		\shandentry{CRB}{Covise Request Broker}
		\shandentry{VRB}{Virtual Reality Request Broker}
		\shandentry{VRBC}{}
	\end{tabular}

	
%	\inputExists{security/security.tex}

	\chapter{Protocols}

	\begin{itemize}
		\item In order to ensure format compatibility, the code has to be compiled with a c++ compiler which implements 'int' as 32-bit IEEE int.
		\begin{itemize}
			\item int's are per definition IEEE int's with 4 bytes
		\end{itemize}
		\item Why should the sender ID not exceed 3 bytes ? Looking at the protocol, there should be 4 bytes available in the protocol (both UDPMessage as well as regular Messages). 
	\end{itemize}

	\chapter{Open Questions}

	\begin{itemize}
		\item General Code Organisation
		\begin{itemize}
			\item What components are in src/sys ?
		\end{itemize}
		\item Format of Data/TB Type ?
		\begin{itemize}
			\item Probably:
			\begin{itemize}
				\item Int32 Length
				\item byte[Length] data in TokenBuffer Format
			\end{itemize}
		\end{itemize}
		\item Determine TokenBuffer Format
		\begin{itemize}
			\item Are type bytes only in Debug Mode Present ?
			\item First byte without Type: TokenBuffer Debug protocol Flag ?
			\item What effects does this have on the protocol ? $\rightarrow$ Are VRB in Debug TB mode and Covise without Debug TB mode able to communicate ?
			\item is the debug byte always present, or is it only in Debug mode present ?
			\item Perhaps redesign to Protocol flags in CONTACT or SET\_USER\_INFO ? Esp. in case debug is not always present. (not just zero but alltogether missing)
			\item Why ????....
		\end{itemize}
	\end{itemize}

	\chapter{Whishlist}
	
	\section{Network Protocol}
	
	\begin{itemize}
		\item Versioning of Protocol, e.g. in initial server connection. Version should be increased with all breaking changes. This way users can be warned if their software uses an old protocol version
		\item Foreward/Backward Compatibility of messages. Changes for this sould probably have to be implemented in Tokenbuffer in form of either an quick index to declare which data is where, or by adding a struct type to serialiser, and then keeping the data layout within a single struct append-only. With the struct type, the fields are append-only within a struct, and not for the whole message. (which could lead to problems using the serialiser for serialising entire objects)
	\end{itemize}

\end{document}